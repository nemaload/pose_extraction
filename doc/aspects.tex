\documentclass[letter,11pt]{article}

\usepackage[english]{babel}
\usepackage[T1]{fontenc}
%\usepackage[ansinew]{inputenc}
\usepackage{lmodern}	% font definition
\usepackage{datetime}
\usepackage{verbatim}

\usepackage[compact]{titlesec}
\usepackage{parskip}
\usepackage{graphicx}
\usepackage[top=2.0cm,bottom=2.0cm,right=2.0cm,left=2.0cm]{geometry}
\usepackage{pdflscape}
\usepackage[numbers]{natbib}

\usepackage[urw-garamond]{mathdesign}

\usepackage{tikz}

%%%<
%\usepackage{verbatim}
%\usepackage[active,tightpage]{preview}
%\PreviewEnvironment{tikzpicture}
%\setlength\PreviewBorder{5pt}%
%%%>

\usetikzlibrary{arrows,shapes}

\usepackage{url}
\usepackage[ps2pdf,breaklinks=true,bookmarks=true,bookmarksopen,bookmarksopenlevel=1,pdfpagelayout=OneColumn,pagebackref=true]{hyperref}
\usepackage{breakurl}
\usepackage{makeidx}

\hypersetup{
    bookmarks=true,         % show bookmarks bar?
    unicode=false,          % non-Latin characters in Acrobat’s bookmarks
    pdftoolbar=false,        % show Acrobat’s toolbar?
    pdfmenubar=true,        % show Acrobat’s menu?
    pdffitwindow=true,     % window fit to page when opened
    pdfstartview={FitV},    % fits the width of the page to the window
    pdftitle={Nemaload},    % title
    pdfauthor={David Dalrymple},     % author
    pdfcreator={SU Project Nemaload Team},   % creator of the document
    pdfproducer={David Dalrymple}, % producer of the document
    pdfnewwindow=true,      % links in new window
    colorlinks=true,       % false: boxed links; true: colored links
    linkcolor=blue,          % color of internal links
    citecolor=green,        % color of links to bibliography
    filecolor=magenta,      % color of file links
    urlcolor=cyan           % color of external links
}

\newcommand{\attrib}[1]{\nopagebreak{\raggedleft\footnotesize #1\par}}
\newcommand{\todo}[1]{\textcolor{lightgray}{\textit{<<#1>>}}}
\newcommand{\tbc}{\begin{center} \todo{to be completed} \end{center}}
\newcommand{\tbcsubsubsection}[1]{ \refstepcounter{subsubsection}%
  \subsubsection*{\thesubsubsection \quad #1} \tbc}
\newcommand{\cel}{{\em C. elegans}}

\setlength{\parskip}{0.3cm plus3mm minus1mm}
\setlength{\parindent}{0cm}

%\titlespacing{\section}{0pt}{0.5cm plus2mm minus3mm}{0.4cm plus3mm minus3mm}

\begin{document}

\begin{center}
  \textsc{\LARGE \textbf{Overview of Techniques} \\[0.5mm] for Various Aspects of the \\[2.8mm] C. elegans Modeling Project}\\[3mm]
	\textit{\Large David Dalrymple}\\[2mm]
  {\large \textbf{version 0.9}, compiled \currenttime, \today\\[0mm]}
	\rule[2mm]{0.66\textwidth}{0.25mm}
\end{center}

\section*{Introduction}

A project as ambitious as realistically emulating the nervous system of an entire organism
necessarily consists of many parts and stages. In addition, in our project, there are
multiple promising technologies that can serve each of these. In this document, I've
identified four main phases, which correspond roughly to the phases of the scientific
method---observation, the collection of data about what is happening in the neurons;
modeling, the synthesis of this data into predictive models of neuronal function;
stimulation, the perturbation of the nervous system so as to collect more nuanced
data about its functional relationships; and finally, verification, the techniques
for determining the accuracy or fitness of the models produced by the coaction
of the other parts of the project.

\tableofcontents

\section{Observation}

\fbox{\textbf{Goal:} Collect neural activation data}

\subsection{Optics}
\label{optics1}

It has been suggested that a spinning-disk confocal microscope is the best
platform for imaging flourescence in individual \cel\ cells.  Other
possibilities include two-photon microscopy and scanning-laser confocal
microscopy. As I know rather little about optics and understand only the basic
principles of operation of these devices, I leave it to others---or better, to
empirical trial---to determine which of these is the most promising.

\subsection{Sensor Molecules}

Many calcium- and voltage-sensitive dyes are commonly referred to in the literature,
including:
\begin{itemize}
  \item \textbf{RH-155} {\em (voltage-sensitive)}
  \item \textbf{RH-414} {\em (voltage-sensitive)}
  \item \textbf{RH-482} {\em (voltage-sensitive)}
  \item \textbf{Cameleon} {\em (calcium-sensitive, ratiometric, commonly used in worms)}
  \item \textbf{GCaMP2} {\em (calcium-sensitive)}
  \item \textbf{GCaMP3} {\em (calcium-sensitive, 3 times brighter than GCaMP2)}
  \item \textbf{GCaMP5} {\em (calcium-sensitive, not shown to work in worms)}
\end{itemize}

For the time being, GCaMP3 is being considered as the leading candidate, but
this may change due to new information or as novel molecules are introduced.%
\footnote{Note: as of this writing (October 5), we are using GFP in place of a
voltage- or calcium- sensitive dye, because such transgenic constructs are
already available, and as a test for the optics and signal separation areas
of this phase.}

\subsection{Signal Separation}

As nearly all of the data collected by these means will consist of images, it will be
necessary to apply some computer vision techniques, at least in the initial stages of
analysis. It may also be necessary to apply some of the genetic approaches discussed
in section \ref{genetics}.

\subsubsection{Straightening}

The first step to processing these images will be to straighten and register the posture
of each animal on a common anterior/posterior/left/right/dorsal/ventral coordinate system.
Parts of this can be done manually, but work is underway to implement the algorithm of
\cite{straighten} and apply it to the preliminary data of October 6.

\subsubsection{Neural Labeling}

In addition, it would be desirable to separate the time-varying signals of each neuron
algorithmically. However, a full labeling of neurons may prove intractable, and depending
on the performance of the straightening algorithm, it may be possible to run modeling
techniques directly on the straightened and registered image data. Intermediate approaches
are also possible.


\section{Modeling}

\fbox{\textbf{Goal:} Estimate functional relationships from experimental data}

This is probably the least developed section in my current thinking. A great deal
of discussion and learning will need to take place before this part of the project is
well characterized. Some initial scattered thoughts are represented below.

\subsection{Correlation Matrix}

A simple way to begin, given time-series data of multiple signals from
separated neurons, is to compute the correlations of each signal with each
other signal, with some selection of time delays. This would elucidate the
simplest of functional relationships (e.g. calcium rising in neuron 1 causes
calcium to rise in neuron 2 100ms later).  It would even be possible to
represent the entire nervous system by a Markov model given this matrix of
correlations. Although I doubt this would produce very biologically meaningful
results, it is a fairly simple first step to try it and see what happens---it
might work better than I expect.

\subsection{Kernel Methods}

However, a correlation matrix would be a very basic analysis of the data. Much
more mathematical investigation is needed, but preliminary research suggests
that kernel methods are most appropriate, due to their resistance to curse of
dimensionality, in addition to the lack of necessity to choose an {\em a
priori} physical model.

%. . .

\subsection{Control Theory}

It may also be possible to use some modeling techniques from control theory in
real time with an optical stimulation setup as described in section \ref{stimulation}
in order to more accurately determine functional relationships.


\section{Stimulation}
\label{stimulation}

\fbox{\textbf{Goal:} Control for certain neural variables (while also collecting data)}

Beyond inferring functional relationships by examining time-series data of calcium activation
under normal \cel\ behavior, we can gather more information by controlling for some of the
variables; that is, by using optogenetics to directly transiently stimulate or inhibit the
activity of specific neurons, and observe the effects on the rest of the nervous system.

\subsection{Opsins}

Optogenetics is the expression of certain opsin proteins in neurons for the purpose of optically
controlling neural activity. The Optogenetics Resource Center lists the following opsins:
\begin{itemize}
  \item \textbf{VChR1}, {\em Volvox} Channelrhodopsin-1 {\em (excitatory; older, obsoleted by ChR2)}
  \item \textbf{NpHR}, Halorhodopsin {\em (inhibitory; known to work in \cel, silences around 100 pA)}
  \item \textbf{ChR2}, Channelrhodopsin-2 {\em (excitatory; known to work in \cel)}
  \item \textbf{hChR2}, humanized ChR2 {\em (excitatory; bistable)}
  \item \textbf{ChETA}, engineered ChR2 {\em (excitatory; very fast)}
  \item \textbf{eNpHR3.0}, third-generation engineered NpHR {\em (inhibitory; silences around 900 pA)}
\end{itemize}
In addition, at least two recent rhodopsins have been developed by the Boyden group:
\begin{itemize}
  \item \textbf{Mac} {\em (inhibitory; responds to blue light)}
  \item \textbf{Arch} {\em (inhibitory; silences around 350 pA)}
\end{itemize}

It is likely best to being using ChR2 and NpHR, since they are known to work in \cel , and speed
is not necessary for our application (the rise time for GCaMP3 is on the order of 100ms, and the
decay time around 600ms). However, as with sensor molecules, this may change as the techniques
develop, or new information is learned.

\subsection{Optics}

It is expected that the optogenetic stimulation will take place through the
same microscopy setup as used for the activation readout (see section
\ref{optics1}). However, to select individual neurons, instead of illuminating
an entire animal uniformly, we may use a similar setup to the DMD system
currently in the Samuel Lab. How this system will be integrated with the
spinning-disk confocal microscope is, as far as I can tell, currently unknown.
It may be possible to do even better than the DMD by taking advantage of the
spinning disk itself. This is a question that should ideally be taken up by
somebody who knows more about optics than I.

\subsection{Genetics}
\label{genetics}

If perfect 3D selectivity in stimulation is not possible optically---as it
probably won't be---we may also use some genetic tools to limit the subset
of cells expressing opsins.

\subsubsection{Genetic Mosaic}

One possibility, suggested by Ed Boyden, is the generation of genetic mosaics:
this is also not a technique that I understand fully, but essentially, some
transgene is introduced in such a way that at every cell division in the lineage,
there is some probability that the transgene will not be copied, and will only
continue down one branch of the lineage. If this procedure is performed sufficiently
many times, and a sufficiently good screen is developed, then in theory, we can
isolate worms expressing opsins only in one neuron at a time, and sort them out
by which neurons are labeled (then we can use these worms as a basis to probe the
effects of the labeled neurons on the rest of the nervous system).

\subsubsection{Identification of Promoters}

An alternative, more traditional approach, is to identify promoters that are specific
to certain limited subsets of cells, in combination with the optical techniques above,
to achieve functional single-cell resolution. (That is, some of the selection can be
done spatially, and some could be done genetically.)


\section{Verification}

\fbox{\textbf{Goal:} Determine the accuracy or fitness of a given model}

``How will you know when you are done?'' is a question I am commonly asked regarding this project.
Below is my current thinking on this topic.

\subsection{Quantifying Behavior}

One possible approach is to make well-known behavioral assays (in chemotaxis,
thermotaxis, etc.) testable algorithmically and quantitatively from vision
tracking data.  We could then reproduce these assays in a virtual environment
with our modeled worms and see that the numbers fall well within the standard
distribution of a population of real wild-type worms.

\subsubsection{Biophysical Simulation}

To do this, we must model not only the nervous system, but also, to a certain degree, both the body and the environment of \cel .
Suzuki and Ohtake \cite{ohtake} have a simplified body model of \cel , as does Lockery \cite{lockery1010}. One of these might form
a good starting point.

\subsection{Predictable Perturbations}

Perhaps a more tractable approach than quantifying the wild-type behavior is to produce predictive results regarding defective animals.

\subsubsection{Mutants}

Several mutants, such as {\em unc-3}, {\em unc-6}, {\em unc-30}, {\em unc-76}, and {\em unc-86}, have well-characterized behavioral defects
in addition to well-characterized defects in neural morphology (the presumptive cause). Ideally, given these defects in neural morphology,
our model should be able to predict the behavioral defects.

\subsubsection{Laser Ablation}

In addition to the relatively small space of genetic mutants with well-known behavior and neural defects, we can also introduce arbitrary
neural defects by killing neurons with laser ablation, and performing behavioral assays to see if the behavior of such animals matches a
prediction by our models.

\subsubsection{Laser Inhibition (Halorhodopsin)}

If we can genetically and/or optically isolate a given neuron for optogenetic
stimulation during behavior (see section \ref{stimulation}), then it may be
more desirable to transiently inhibit the activity of such a neuron than to
kill it, for the purpose of generating more nuanced data to be matched against
a model's prediction.

\end{document}
