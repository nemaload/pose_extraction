\documentclass[letter,11pt]{article}

\usepackage[english]{babel}
\usepackage[T1]{fontenc}
%\usepackage[ansinew]{inputenc}
\usepackage{lmodern}	% font definition
\usepackage{datetime}
\usepackage{verbatim}

\usepackage{parskip}
\usepackage{graphicx}
\usepackage[top=2.2cm,bottom=2.1cm,right=1.9cm,left=1.9cm]{geometry}
\usepackage{pdflscape}
\usepackage[numbers]{natbib}

\usepackage[urw-garamond]{mathdesign}

\usepackage{tikz}

%%%<
%\usepackage{verbatim}
%\usepackage[active,tightpage]{preview}
%\PreviewEnvironment{tikzpicture}
%\setlength\PreviewBorder{5pt}%
%%%>

\usetikzlibrary{arrows,shapes}

\usepackage{url}
\usepackage[ps2pdf,breaklinks=true,bookmarks=true,bookmarksopen,bookmarksopenlevel=1,pdfpagelayout=OneColumn,pagebackref=true]{hyperref}
\usepackage{breakurl}
\usepackage{makeidx}

\hypersetup{
    bookmarks=true,         % show bookmarks bar?
    unicode=false,          % non-Latin characters in Acrobat’s bookmarks
    pdftoolbar=false,        % show Acrobat’s toolbar?
    pdfmenubar=true,        % show Acrobat’s menu?
    pdffitwindow=true,     % window fit to page when opened
    pdfstartview={FitV},    % fits the width of the page to the window
    pdftitle={Nemaload},    % title
    pdfauthor={David Dalrymple},     % author
    pdfcreator={SU Project Nemaload Team},   % creator of the document
    pdfproducer={David Dalrymple}, % producer of the document
    pdfnewwindow=true,      % links in new window
    colorlinks=true,       % false: boxed links; true: colored links
    linkcolor=blue,          % color of internal links
    citecolor=green,        % color of links to bibliography
    filecolor=magenta,      % color of file links
    urlcolor=cyan           % color of external links
}

\renewcommand*{\backref}[1]{}
\renewcommand*{\backrefalt}[4]{%
  \ifcase #1 %
(Not cited.)%
\or
(Cited on page #2.)%
\else
(Cited on pages #2.)%
\fi
}

\newcommand{\attrib}[1]{\nopagebreak{\raggedleft\footnotesize #1\par}}
\newcommand{\todo}[1]{\textcolor{lightgray}{\textit{<<#1>>}}}
\newcommand{\tbc}{\begin{center} \todo{to be completed} \end{center}}
\newcommand{\tbcsubsubsection}[1]{ \refstepcounter{subsubsection}%
  \subsubsection*{\thesubsubsection \quad #1} \tbc}
\newcommand{\cel}{{\em C. elegans}}

\setlength{\parskip}{0.3cm plus3mm minus1mm}
\setlength{\parindent}{0cm}

\begin{document}

\begin{center}
	\textsc{\LARGE ``{\bfseries C. elegant}''}\\[1mm]
	\textit{David Dalrymple}\\[1mm]
	%{\small {\bfseries \color{red} Draft 3} compiled \currenttime, \today\\[0mm]}
	\rule[2mm]{0.66\textwidth}{0.25mm}\\[10mm]
\end{center}

\section{Introduction}

We propose a novel program to fully model the relationship between the neural
circuits and behavioral dispositions of the organism {\em Caenorhabditis
elegans} (\cel).  This program is well matched to the current level of
technological development, and is poised to provide important basic insights
into systems neuroscience, with likely implications for artificial intelligence
research in the future. As a fundamental scientific outcome, it provides a
unique opportunity to establish an upper bound on the level of detail in neural
simulation that is necessary to make predictions about the behavior of an
entire organism; as a side product, the project may also contribute some degree
of understanding about intermediate levels of abstraction between neurons and
organism behavior.

\section{Initial {\em in silico} work}
\label{initial}

We will begin with the closest current result to our eventual goal, a 2004
paper by Suzuki and Ohtake \cite{ohtake} in which 18 \cel\ neurons involved in
gentle touch response were modeled, using a real-coded genetic algorithm to
tune the unknown parameters of a very simple sigmoidal neuron model to a
predetermined mathematical model of the expected system behavior. There are a
number of improvements that can be made to this approach immediately: using a
more principled optimization technique, a more sophisticated model of behavior,
and incorporating more interneurons in the model.

\section{Behavioral and environmental modeling}

Since our goal is to replicate the behavior of an organism---its interaction with
its environment---a critical component of the project is to accurately model the
environment and develop a quantitative assessment of the behaviors of interest.
Fortunately, the environment in which \cel\ is usually observed is quite simple
(a dish of agar). However, a literature search and possibly some new behavioral
experiments will be necessary to establish a quantitative description of \cel\ 
behavior.

\section{Optimization and meta-optimization}

With a neural topology and a quantitative description of behavior, quite a bit
of new work can be done using global optimization techniques to determine the
parameters of a neural model to match the behaviors described. However, it is
not necessarily the case that these parameters would have any biological
meaning. Thus, some physical experiments on \cel\ are necessary to source this
information. However, since most combinations of parameters would, in simulation,
produce dramatically different behavior from that expected, it may be possible to
apply a ``meta-optimization'' technique to generate experiments that would 
provide the most information about which set of parameters represents biological
reality.

\section{Experimental technologies}

Such a project as this would not have been feasible five years ago. Without the
tools to directly probe functional relationships between neurons, any computer-
generated theories about how behavior emerges from neural circuits would
essentially be guesses. This project involves significant biological work, and
the experimental tools that enable such work are briefly discussed below.

\subsection{Optogenetics}

In 2005, Ed Boyden and collaborators published \cite{boyden} a technique for
optical control of neural potential through transgenic rhodopsins. It is now
possible to stimulate or inhibit individual neurons through a purely optical
experimental setup. This has enabled a wide variety of neuroscientific research
that previously would have been impractical.

\subsection{Calcium dyes}

As optogenetics provide optical input to neural circuits, calcium dyes such as
G-CaMP3 provide optical output from neural circuits by activating flourescent
proteins when calcium is present. The Ramanathan lab at Harvard has
demonstrated \cite{ramanathan} simultaneous stimulation via optogenetics and
readout via calcium dye, in \cel, to determine the functional relationship of
certain neurons to other neurons.

\subsection{Genetic mosaic}

Although the Samuel lab at Harvard has developed \cite{leifer} a DLP-based
targeting system for optically stimulating specific neurons, it is difficult
to prevent this stimulation from affecting nearby neurons with optogenetic
proteins expresed.  One possible workaround is the FLP/FRT system of genetic
mosaics \cite{mosaic}, which causes the expression of a transgene in the
descendants of only one cell during the development of the organism. It would
be possible to sort out those mosaics in which exactly one neuron is labeled,
and perform experiments on these to determine the effect of stimulating that
single neuron on all the other neurons in the organism.

\section{Neuron model}

One of the significant open questions into which this work may provide insight
is the level of detail at which neurons themselves must be modeled at to
produce organism-level behavioral equivalence. We will begin by assuming a
simple sigmoidal neuron model, as in \cite{ohtake}, but we will later
experiment with varying the level of detail, both to a lower biophysical level
and to attempt higher-level abstractions on top of the neurons.

\section{Conclusion}

In conclusion, we believe that now is the right time to begin a major project
to understand how a biological nervous system gives rise to its organism's
behavior, and that the nematode \cel\ is the right organism to begin with,
thanks to a confluence of traditional experimental data, tractable system
complexity, and exciting new technologies for manipulating neural systems.

\phantomsection
\addcontentsline{toc}{section}{References}
\pretolerance=80
\nocite{*}
\bibliographystyle{abbrvnat}
\bibliography{proposal}

\end{document}
