\documentclass[letter,11pt]{article}

\usepackage[english]{babel}
\usepackage[T1]{fontenc}
%\usepackage[ansinew]{inputenc}
\usepackage{lmodern}	% font definition
\usepackage{datetime}
\usepackage{verbatim}

\usepackage{parskip}
\usepackage{graphicx}
\usepackage[top=2.2cm,bottom=2.2cm,right=1.9cm,left=1.9cm]{geometry}
\usepackage{pdflscape}
\usepackage[numbers]{natbib}

\usepackage[urw-garamond]{mathdesign}

\usepackage{tikz}

%%%<
%\usepackage{verbatim}
%\usepackage[active,tightpage]{preview}
%\PreviewEnvironment{tikzpicture}
%\setlength\PreviewBorder{5pt}%
%%%>

\usetikzlibrary{arrows,shapes}

\usepackage{url}
\usepackage[ps2pdf,breaklinks=true,bookmarks=true,bookmarksopen,bookmarksopenlevel=1,pdfpagelayout=OneColumn,pagebackref=true]{hyperref}
\usepackage{breakurl}
\usepackage{makeidx}

\hypersetup{
    bookmarks=true,         % show bookmarks bar?
    unicode=false,          % non-Latin characters in Acrobat’s bookmarks
    pdftoolbar=false,        % show Acrobat’s toolbar?
    pdfmenubar=true,        % show Acrobat’s menu?
    pdffitwindow=true,     % window fit to page when opened
    pdfstartview={FitV},    % fits the width of the page to the window
    pdftitle={Nemaload},    % title
    pdfauthor={David Dalrymple},     % author
    pdfcreator={SU Project Nemaload Team},   % creator of the document
    pdfproducer={David Dalrymple}, % producer of the document
    pdfnewwindow=true,      % links in new window
    colorlinks=true,       % false: boxed links; true: colored links
    linkcolor=blue,          % color of internal links
    citecolor=green,        % color of links to bibliography
    filecolor=magenta,      % color of file links
    urlcolor=cyan           % color of external links
}

\newcommand{\attrib}[1]{\nopagebreak{\raggedleft\footnotesize #1\par}}
\newcommand{\todo}[1]{\textcolor{lightgray}{\textit{<<#1>>}}}
\newcommand{\tbc}{\begin{center} \todo{to be completed} \end{center}}
\newcommand{\tbcsubsubsection}[1]{ \refstepcounter{subsubsection}%
  \subsubsection*{\thesubsubsection \quad #1} \tbc}
\newcommand{\cel}{{\em C. elegans}}

\setlength{\parskip}{0.3cm plus3mm minus1mm}
\setlength{\parindent}{0cm}

\begin{document}

\begin{center}
  \textsc{\LARGE \textbf{Overview of Techniques} \\[0.5mm] for Various Aspects of the \\[2.8mm] C. elegans Modeling Project}\\[3mm]
	\textit{\Large David Dalrymple}\\[2mm]
	{\large compiled \currenttime, \today\\[0mm]}
	\rule[2mm]{0.66\textwidth}{0.25mm}
\end{center}

\section*{Introduction}

A project as ambitious as realistically emulating the nervous system of an entire organism
necessarily consists of many parts and stages. In addition, in our project, there are
multiple promising technologies that can serve each of these. In this document, I've
identified four main areas, which correspond roughly to the phases of the scientific
method---observation, the collection of data about what is happening in the neurons;
modeling, the synthesis of this data into predictive models of neuronal function;
stimulation, the perturbation of the nervous system so as to collect more nuanced
data about its functional relationships; and finally, verification, the techniques
for determining the accuracy or fitness of the models produced by the coaction
of the other parts of the project.

\tableofcontents

\section{Observation}

\subsection{Optics}

It has been suggested that a spinning-disk confocal microscope is the best
platform for imaging flourescence in individual \cel\ cells.  Other
possibilities include two-photon microscopy and scanning-laser confocal
microscopy. As I know rather little about optics and understand only the basic
principles of operation of these devices, I leave it to others---or better, to
empirical trial---to determine which of these is the most promising.

\subsection{Sensor Molecules}

Many calcium- and voltage-sensitive dyes are commonly referred to in the literature,
including:
\begin{itemize}
\item RH-155 {\em (voltage-sensitive)}
\item RH-414 {\em (voltage-sensitive)}
\item RH-482 {\em (voltage-sensitive)}
\item Cameleon {\em (calcium-sensitive, ratiometric, commonly used in worms)}
\item G-CaMP2 {\em (calcium-sensitive)}
\item G-CaMP3 {\em (calcium-sensitive, 3 times brighter than G-CaMP2)}
\item G-CaMP5 {\em (calcium-sensitive, not shown to work in worms)}
\end{itemize}

For the time being, G-CaMP3 is being considered as the leading candidate, but this
may change due to new information or as novel molecules are introduced.

\subsection{Image Analysis}

As nearly all of the data collected by these means will consist of images, it will be
necessary to apply some computer vision techniques, at least in the initial stages of
analysis.

\subsubsection{Straightening}

The first step to processing these images will be to straighten and register the posture
of each animal on a common anterior/posterior/left/right/dorsal/ventral coordinate system.
Parts of this can be done manually, but work is underway to implement the algorithm of
\cite{straighten} and apply it to the preliminary data of October 6.

\subsubsection{Signal Separation}

In addition, it would be desirable to separate the time-varying signals of each neuron
algorithmically. However, a full labeling of neurons may prove intractable, and depending
on the performance of the straightening algorithm, it may be possible to run modeling
techniques directly on the straightened and registered image data. Intermediate approaches
are also possible.


\section{Modeling}

This is probably the least developed section in my current thinking. A great deal
of discussion and learning will need to take place before this part of the project is
well characterized. Some initial scattered thoughts are represented below.

\subsection{Correlation Matrix}

\subsection{Kernel Methods}

%. . .

\subsection{Control Theory}



\section{Stimulation}
\label{stimulation}

\subsection{Rhodopsins}

\subsection{Optics}

\subsection{Genetic Mosaic}


\section{Verification}

``How will you know when you are done?'' is a question I am commonly asked regarding this project.
Below is my current thinking on this subject.

\subsection{Quantifying Behavior}

One possible approach is to make well-known behavioral assays (in chemotaxis,
thermotaxis, etc.) testable algorithmically and quantitatively from vision
tracking data.  We could then reproduce these assays in a virtual environment
with our modeled worms and see that the numbers fall well within the standard
distribution of a population of real wild-type worms.

\subsubsection{Biophysical Simulation}

To do this, we must model not only the nervous system, but also, to a certain degree, both the body and the environment of \cel .
Suzuki and Ohtake \cite{ohtake} have a simplified body model of \cel , as does Lockery \cite{lockery1010}. One of these might form
a good starting point.

\subsection{Predictable Perturbations}

Perhaps a more tractable approach than quantifying the wild-type behavior is to produce predictive results regarding defective animals.

\subsubsection{Mutants}

Several mutants, such as {\em unc-3}, {\em unc-6}, {\em unc-30}, {\em unc-76}, and {\em unc-86}, have well-characterized behavioral defects
in addition to well-characterized defects in neural morphology (the presumptive cause). Ideally, given these defects in neural morphology,
our model should be able to predict the behavioral defects.

\subsubsection{Laser Ablation}

In addition to the relatively small space of genetic mutants with well-known behavior and neural defects, we can also introduce arbitrary
neural defects by killing neurons with laser ablation, and performing behavioral assays to see if the behavior of such animals matches a
prediction by our models.

\subsubsection{Laser Inhibition (Halorhodopsin)}

If we can genetically and/or optically isolate a given neuron for optogenetic
stimulation during behavior (see section \ref{stimulation}), then it may be
more desirable to transiently inhibit the activity of such a neuron than to
kill it, for the purpose of generating more nuanced data to be matched against
a model's prediction.

\end{document}
