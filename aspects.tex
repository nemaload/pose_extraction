\documentclass[letter,11pt]{article}

\usepackage[english]{babel}
\usepackage[T1]{fontenc}
%\usepackage[ansinew]{inputenc}
\usepackage{lmodern}	% font definition
\usepackage{datetime}
\usepackage{verbatim}

\usepackage{parskip}
\usepackage{graphicx}
\usepackage[top=2.2cm,bottom=2.2cm,right=1.9cm,left=1.9cm]{geometry}
\usepackage{pdflscape}
\usepackage[numbers]{natbib}

\usepackage[urw-garamond]{mathdesign}

\usepackage{tikz}

%%%<
%\usepackage{verbatim}
%\usepackage[active,tightpage]{preview}
%\PreviewEnvironment{tikzpicture}
%\setlength\PreviewBorder{5pt}%
%%%>

\usetikzlibrary{arrows,shapes}

\usepackage{url}
\usepackage[ps2pdf,breaklinks=true,bookmarks=true,bookmarksopen,bookmarksopenlevel=1,pdfpagelayout=OneColumn,pagebackref=true]{hyperref}
\usepackage{breakurl}
\usepackage{makeidx}

\hypersetup{
    bookmarks=true,         % show bookmarks bar?
    unicode=false,          % non-Latin characters in Acrobat’s bookmarks
    pdftoolbar=false,        % show Acrobat’s toolbar?
    pdfmenubar=true,        % show Acrobat’s menu?
    pdffitwindow=true,     % window fit to page when opened
    pdfstartview={FitV},    % fits the width of the page to the window
    pdftitle={Nemaload},    % title
    pdfauthor={David Dalrymple},     % author
    pdfcreator={SU Project Nemaload Team},   % creator of the document
    pdfproducer={David Dalrymple}, % producer of the document
    pdfnewwindow=true,      % links in new window
    colorlinks=true,       % false: boxed links; true: colored links
    linkcolor=blue,          % color of internal links
    citecolor=green,        % color of links to bibliography
    filecolor=magenta,      % color of file links
    urlcolor=cyan           % color of external links
}

\newcommand{\attrib}[1]{\nopagebreak{\raggedleft\footnotesize #1\par}}
\newcommand{\todo}[1]{\textcolor{lightgray}{\textit{<<#1>>}}}
\newcommand{\tbc}{\begin{center} \todo{to be completed} \end{center}}
\newcommand{\tbcsubsubsection}[1]{ \refstepcounter{subsubsection}%
  \subsubsection*{\thesubsubsection \quad #1} \tbc}
\newcommand{\cel}{{\em C. elegans}}

\setlength{\parskip}{0.3cm plus3mm minus1mm}
\setlength{\parindent}{0cm}

\begin{document}

\begin{center}
  \textsc{\LARGE \textbf{Overview of Techniques} \\[0.5mm] for Various Aspects of the \\[2.8mm] C. elegans Modeling Project}\\[3mm]
	\textit{\Large David Dalrymple}\\[2mm]
	{\large compiled \currenttime, \today\\[0mm]}
	\rule[2mm]{0.66\textwidth}{0.25mm}
\end{center}

\section*{Introduction}

A project as ambitious as realistically emulating the nervous system of an entire organism
necessarily consists of many parts and stages. In addition, in our project, there are
multiple promising technologies that can serve each of these. In this document, I've
identified four main areas, which correspond roughly to the phases of the scientific
method---observation, the collection of data about what is happening in the neurons;
modeling, the synthesis of this data into predictive models of neuronal function;
stimulation, the perturbation of the nervous system so as to collect more nuanced
data about its functional relationships; and finally, verification, the techniques
for determining the accuracy or fitness of the models produced by the coaction
of the other parts of the project.

\tableofcontents

\section{Observation}

\subsection{Sensor Molecules}

\subsection{Optics}



\subsection{Image Analysis}

\subsubsection{Straightening}

\subsubsection{Signal Separation}



\section{Modeling}

\subsection{Correlation Matrix}

\subsection{Kernel Methods}

%. . .

\subsection{Control Theory}



\section{Stimulation}

\subsection{Rhodopsins}

\subsection{Optics}



\section{Verification}

\subsection{Quantifying Behavior}

\subsection{Predictable Perturbations}

\subsubsection{Mutants}

\subsubsection{Laser Ablation}

\subsubsection{Laser Inhibition (Halorhodopsin)}

\subsection{Biophysical Simulation}



\end{document}
