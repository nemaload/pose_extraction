\documentclass[letter,11pt]{article}

\usepackage[english]{babel}
\usepackage[T1]{fontenc}
%\usepackage[ansinew]{inputenc}
\usepackage{lmodern}	% font definition
\usepackage{verse}
\usepackage{datetime}
\usepackage{verbatim}

\usepackage{parskip}
\usepackage{graphicx}
\usepackage[top=2.2cm,bottom=2.1cm,right=1.9cm,left=1.9cm]{geometry}
\usepackage{pdflscape}
\usepackage[numbers]{natbib}

\usepackage[urw-garamond]{mathdesign}

\usepackage{tikz}

%%%<
%\usepackage{verbatim}
%\usepackage[active,tightpage]{preview}
%\PreviewEnvironment{tikzpicture}
%\setlength\PreviewBorder{5pt}%
%%%>

\usetikzlibrary{arrows,shapes}

\usepackage{url}
\usepackage[ps2pdf,breaklinks=true,bookmarks=true,bookmarksopen,bookmarksopenlevel=1,pdfpagelayout=OneColumn,pagebackref=true]{hyperref}
\usepackage{breakurl}
\usepackage{makeidx}

\hypersetup{
    bookmarks=true,         % show bookmarks bar?
    unicode=false,          % non-Latin characters in Acrobat’s bookmarks
    pdftoolbar=false,        % show Acrobat’s toolbar?
    pdfmenubar=true,        % show Acrobat’s menu?
    pdffitwindow=true,     % window fit to page when opened
    pdfstartview={FitV},    % fits the width of the page to the window
    pdftitle={Nemaload},    % title
    pdfauthor={David Dalrymple},     % author
    pdfcreator={SU Project Nemaload Team},   % creator of the document
    pdfproducer={David Dalrymple}, % producer of the document
    pdfnewwindow=true,      % links in new window
    colorlinks=true,       % false: boxed links; true: colored links
    linkcolor=blue,          % color of internal links
    citecolor=green,        % color of links to bibliography
    filecolor=magenta,      % color of file links
    urlcolor=cyan           % color of external links
}

\newcommand{\attrib}[1]{\nopagebreak{\raggedleft\footnotesize #1\par}}
\newcommand{\todo}[1]{\textcolor{lightgray}{\textit{<<#1>>}}}
\newcommand{\tbc}{\begin{center} \todo{to be completed} \end{center}}
\newcommand{\tbcsubsubsection}[1]{ \refstepcounter{subsubsection}%
  \subsubsection*{\thesubsubsection \quad #1} \tbc}
\newcommand{\cel}{{\em C. elegans}}

\setlength{\parskip}{0.3cm plus3mm minus1mm}
\setlength{\parindent}{0cm}

\begin{document}

\begin{center}
	\textsc{\LARGE \bfseries C. elegant}\\[1mm]
	\textit{David Dalrymple}\\[1mm]
	{\small {\bfseries \color{red} Draft 1} compiled \currenttime, \today\\[0mm]}
	\rule[2mm]{0.66\textwidth}{0.25mm}\\[10mm]
\end{center}

\section*{Introduction}

We propose a novel program to fully model the relationship between the neural circuits and behavioral dispositions of the organism {\em Caenorhabditis elegans} (\cel).
This program is well matched to the current level of technological development, and is poised to provide important basic insights into systems neuroscience, with
likely implications for artificial intelligence research in the future. At least, it provides a unique opportunity to establish an upper bound on the level
of detail in neural simulation that is necessary to make predictions about the behavior of an entire organism; as a side product, the project may also contribute
some degree of understanding about intermediate levels of abstraction between neurons and organism behavior.

\section*{Initial {\em in silico} work}

We will begin with the closest current result to our eventual goal, a 2004
paper by Suzuki and Ohtake in which 18 \cel\ neurons involved in gentle touch
response were modeled, using a real-coded genetic algorithm to tune the unknown
parameters of a very simple sigmoidal neuron model to a predetermined mathematical
model of the expected system behavior. There are a number of improvements that
can be made to this approach immediately: using a more principled optimization
technique, a more sophisticated model of behavior, and incorporating more
interneurons in the model.

\section*{Behavioral and environmental modeling}

Since our goal is to replicate the behavior of an organism---its interaction with
its environment---a critical component of the project is to accurately model the
environment and develop a quantitative assessment of the behaviors of interest.
Fortunately, the environment in which \cel\ is usually observed is quite simple
(a dish of agar). However, a literature search and possibly some new behavioral
experiments will be necessary to establish a quantitative description of \cel\ 
behavior.

\section*{Experimental technologies}

Such a project as this would not have been feasible five years ago. Without the
tools to directly probe functional relationships between neurons, any computer-
generated theories about how behavior emerges from neural circuits would
essentially be guesses. This project involves significant biological work, and
the experimental tools that enable such work are briefly discussed below.

\subsection*{Optogenetics}

In 2005, Ed Boyden and collaborators published a technique for optical control of
neural potential, through transgenic rhodopsins. It is now possible to stimulate
or inhibit individual neurons through a purely optical experimental setup. This
has enabled a wide variety of neuroscientific research that previously would have

\subsection*{Calcium dyes}

\subsection*{Genetic mosaic}

\subsection*{Motion tracking}

\section*{Optimization and meta-optimization}

\end{document}
